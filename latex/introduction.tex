\section{Introduction}

Abstract interpretation is a theory of sound approximation.
%
However, most of available abstract interpreters do not formally
establish a relation between their algorithmic theory and
implementations.
%
Several abstract interpreters have been proven correct. The most
notable one is Verasco~\cite{POPL15:Jourdan:al}, a static analyser of C
programs that has been entirely written, specified and proved in the
proof assistant Coq.
%
However, understanding the implementation and proof of Verasco
requires an expertise with Coq and proof assistants.

Proofs in Coq are achieved thanks to extensive use of proof scripts,
that are very difficult for non expert to read.
%
By contrast with a handwritten proof, a Coq proof can be very verbose,
and often does not convey a good intuition for the idea behind a
proof.
%
Thus, writing and proving sound a static analyzer is a complex and
time-consuming task: for example, Verasco requires about 17k
lines~\cite{POPL15:Jourdan:al} of manual Coq proofs.
%
Such an effort, however, yields the strongest guarantees and provides
complete trust in the static analyser.

This paper showcases the implementation of a sound static
analyser using the general-purpose functional programming language
\gls{fstar}.
%
Equipped with dependent types and built-in SMT solver facilities,
\gls{fstar} provides both an OCaml-like experience and proof assistant
capacities. It recently shined with the Project
Everest~\cite{project-everest}, which delivered a series of formally
verified, high-performance, cryptographic libraries:
%
HACL*~\cite{haclstar}, ValeCrypt~\cite{bond2017vale} and
EverCrypt~\cite{protzenko2020evercrypt}; that are for instance used and
deployed in Mozilla Firefox.
%
While \gls{fstar} can always resort to proof scripts similar to Coq's
ones, most proof obligations in \gls{fstar} are automatically
discharged by the SMT solver Z3~\cite{z3}.

We present an abstract interpreter equipped with the numerical
abstract domain of intervals, forward and backward analyses of
expressions, widening, and syntax-directed loop iteration.
%
This paper makes the following contributions.

\begin{itemize}
\item It demonstrates the ease of use of \gls{fstar} for verified
  static analysis: we implement a verified abstract interpreter, and
  show about 95\% of its 527 lines of code (proof included) directly
  in the paper.
  %
\item
  As far as we know, it is the first time SMT techniques are used for
  verifying an abstract interpreter.
  %
\item
  We gain an order of magnitude in the number of proof lines in
  comparison with similar works implemented in Coq.
  %
  %% Our implementation requires proportionally at least ten times less
  %% proof lines\ref{abint:section:conclusion} in comparison.
  %

\end{itemize}



\paragraph{Related work} Efforts in verified abstract interpretation are
numerous~\cite{Pich:these,ITP10:Cachera:Pichardie,SAS13:Blazy:al,Nipkow-ITP12},
and go up to Verasco~\cite{POPL15:Jourdan:al}, a modular, real-world
abstract interpreter verified in Coq.
%
Blazy et al.~\cite{SAS13:Blazy:al} and Verasco follow closely the
modular design of Astrée~\cite{astree}; we exhibit a similar modularity
on a smaller scale.
%
However, such analysers require a non-trivial amount of mechanized
proofs: in constrast, this paper shows that implementing a formally
verified abstract interpreter with very little manual proofs is
possible.
%
So far, verified abstract interpreters have been focused on
concretization-based formalizations.
%
The work of Darais et al.~\cite{darais-oopsla15} is the only one to
really consider the use of galois connections.
%
They provide a minimalist abstract inteperter for imperative language
but this interpreter seems very limited compared to ours.
%
They use the Agda proof assistant which is comparable to Coq in terms
of proof verbosity.

\paragraph{Overview} Section~\ref{abint:section:lang-def} defines $\AbIntImp{}$, the language our
abstract interpreter deals with, to which is given an operational
semantics in Section~\ref{abint:section:operational-sematics}.
%
Then Section~\ref{abint:section:abstract-domains} formalizes lattices and abstract
domains, while Section~\ref{abint:section:intervals} instantiates them with the
abstract domain of intervals.
%
Section~\ref{abint:section:specialized-adoms} derives more specific abstract domains,
for numeric expressions and for memories.
%
The latter is instantiated by Section~\ref{abint:section:weakly-rel-amem}, that implements an
abstract weakly-relational memory.
%
Finally, Section~\ref{abint:section:stmt-ab-int} presents the abstract interpretation
of $\AbIntImp{}$ statements.

\paragraph{}
The \gls{fstar} development is available on
GitHub\footnote{\url{https://github.com/W95Psp/verified-abstract-interpreter}}.
%
The resulting analyser is available online as a web application at
\url{https://w95psp.github.io/verified-abstract-interpreter}.

